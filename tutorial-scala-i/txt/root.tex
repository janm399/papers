\documentclass[10 pt]{article}
\documentclass[10 pt, twocolumn]{article}

\usepackage[utf8]{inputenc}
\usepackage[T1]{fontenc}
\usepackage{caption}
\usepackage{graphicx}
\usepackage{xcolor}
\usepackage{interval}
\usepackage{listingsutf8}
\usepackage{hyperref}
\usepackage{siunitx}
\usepackage{algorithm2e}
\usepackage{rotating}
\usepackage{adjustbox}
\usepackage{booktabs}
\usepackage{pgfplots}
\usepackage{tikz}
\usepackage{footmisc}
\usepackage{amsfonts}
\usepackage[backend=biber,style=numeric]{biblatex}
\usepackage[
  left=1.50cm,
  right=1.50cm,
  top=2.00cm,
  bottom=2.00cm
]{geometry}

\pgfplotsset{compat=1.15}

\sisetup{load-configurations=abbreviations, binary-units=true}
\intervalconfig {
    soft open fences ,
    separator symbol =; ,
}

\lstdefinelanguage{JavaScript}{
  keywords={break, case, catch, continue, debugger, default, delete, do, else, finally, for, function, if, in, instanceof, new, return, switch, this, throw, try, typeof, var, void, while, with},
  morecomment=[l]{//},
  morecomment=[s]{/*}{*/},
  morestring=[b]',
  morestring=[b]",
  sensitive=true
}

\lstdefinelanguage{Protobuf}{
  keywords={message, string, uint32, int32},
  morecomment=[l]{//},
  morecomment=[s]{/*}{*/},
  morestring=[b]',
  morestring=[b]",
  sensitive=true
}

\lstset{
    language=C,
    keywordstyle={\bfseries},
    basicstyle=\footnotesize,
    literate={->}{$\rightarrow{}$}{1} {<-}{$\leftarrow{}$}{1},
    stringstyle=\color{purple},
    keepspaces=true,
    captionpos=b,
    inputencoding=utf8,
    escapeinside={\%*}{*)}
}

\renewcommand\AlCapSty{\text}
\SetAlCapNameFnt{\footnotesize}
\SetAlCapFnt{\footnotesize}
\SetAlgoCaptionSeparator{.}
\DeclareCaptionLabelFormat{nospace}{#1 #2}
\captionsetup[table]{labelformat=nospace,labelfont=rm,name=Table,labelsep=period}

\newcommand{\pcode}[1]{
    \lstinline[basicstyle=\itshape,keywordstyle={}]{#1}
}

\newcommand*{\lstnumberautorefname}{line}

\newcommand{\icode}[1]{\lstinline{#1}}

\newcommand{\name}[1] {\emph{#1}}

\newcolumntype{R}[2]{%
    >{\adjustbox{angle=#1,lap=\width-(#2/2)}\bgroup}%
    l%
    <{\egroup}%
}
\newcommand*\rot{\multicolumn{1}{R{90}{1em}}}

\newcommand{\fig}[3]{
  \begin{figure}[h]
    \begin{center}
        \includegraphics[width=8cm,keepaspectratio]{#1}
        \caption{#3}
        \label{fig:#2}
    \end{center}
  \end{figure}
}

\usepackage[most]{tcolorbox}
\newcounter{example}
\usepackage{xparse}
\usepackage{lipsum}

\def\exampletext{Example} % If English
\NewDocumentEnvironment{example}{ O{} }
{
\colorlet{colexam}{red!55!black} % Global example color
\newtcolorbox[use counter=example]{examplebox}{%
    % Example Frame Start
    empty,% Empty previously set parameters
    title={#1},% use \thetcbcounter to access the example counter text
    % Attaching a box requires an overlay
    attach boxed title to top left,
       % Ensures proper line breaking in longer titles
       minipage boxed title,
    % (boxed title style requires an overlay)
    boxed title style={empty,size=minimal,toprule=0pt,top=4pt,left=3mm,overlay={}},
    coltitle=colexam,fonttitle=\bfseries,
    before=\par\medskip\noindent,parbox=false,boxsep=0pt,left=3mm,right=0mm,top=2pt,breakable,pad at break=0mm,
       before upper=\csname @totalleftmargin\endcsname0pt, % Use instead of parbox=true. This ensures parskip is inherited by box.
    % Handles box when it exists on one page only
    overlay unbroken={\draw[colexam,line width=.5pt] ([xshift=-0pt]title.north west) -- ([xshift=-0pt]frame.south west); },
    % Handles multipage box: first page
    overlay first={\draw[colexam,line width=.5pt] ([xshift=-0pt]title.north west) -- ([xshift=-0pt]frame.south west); },
    % Handles multipage box: middle page
    overlay middle={\draw[colexam,line width=.5pt] ([xshift=-0pt]frame.north west) -- ([xshift=-0pt]frame.south west); },
    % Handles multipage box: last page
    overlay last={\draw[colexam,line width=.5pt] ([xshift=-0pt]frame.north west) -- ([xshift=-0pt]frame.south west); },%
    }
\begin{examplebox}}
{\end{examplebox}\endlist}

\newcommand*\circled[1]{\tikz[baseline=(char.base)]{
            \node[shape=circle,draw,inner sep=1pt] (char) {#1};}}

\usepackage{tikz}

\newcommand{\smiley}{\tikz[baseline=-0.75ex,black]{
    \draw circle (2mm);
\node[fill,circle,inner sep=0.5pt] (left eye) at (135:0.8mm) {};
\node[fill,circle,inner sep=0.5pt] (right eye) at (45:0.8mm) {};
\draw (-145:0.9mm) arc (-120:-60:1.5mm);
    }
}

\newcommand{\frownie}{\tikz[baseline=-0.75ex,black]{
    \draw circle (2mm);
\node[fill,circle,inner sep=0.5pt] (left eye) at (135:0.8mm) {};
\node[fill,circle,inner sep=0.5pt] (right eye) at (45:0.8mm) {};
\draw (-145:0.9mm) arc (120:60:1.5mm);
    }
}

\newcommand{\neutranie}{\tikz[baseline=-0.75ex,black]{
    \draw circle (2mm);
\node[fill,circle,inner sep=0.5pt] (left eye) at (135:0.8mm) {};
\node[fill,circle,inner sep=0.5pt] (right eye) at (45:0.8mm) {};
\draw (-135:0.9mm) -- (-45:0.9mm);
    }
}


\usepackage[OT1]{fontenc}
\newcommand*\eiadfamily{\fontencoding{OT1}\fontfamily{eiad}\selectfont}

\addbibresource{root.bib}

\title{Scala Tutorial I}

\author{Jan Macháček}

\begin{document}

\maketitle
\begin{abstract}
  Scala is a fusion langauge that combines functional and object-oriented programming paradigms in a syntax that is similar to most other \emph{C-like} languages. The ...
\end{abstract}
\bigskip

\section{Zero to hundred}
FizzBuzz is a typical program that follows \emph{Hello, world}, adding iteration and conditions. The Scala version of FizzBuzz is shown in \autoref{code:fb1}--it shows the definition of a function \pcode{def}, followed by name and arguments, and its implementation that follows the \pcode{=} sign. The loop (\pcode{for}) and condition (\pcode{if}, \pcode{else}) keywords are the old friends from other languages. 

\begin{lstlisting}[caption={Fizz Buzz}, label={code:fb1}, language=Scala, escapechar=|]
def fizzBuzz = {
  for (i <- 1 to 100) {
    if (i % 15 == 0) println("FizzBuzz")
    else if (i % 3 == 0) println("Fizz")
    else if (i % 5 == 0) println("Buzz")
    else println(i)
  }
}
\end{lstlisting}

The FizzBuzz from \autoref{code:fb1} isn't particularly re-usable: it simply prints 100 elements to the standard output, nothing else and nothing more. There is no way, for example, to direct the output to a web socket, or to use it to determine how it maps of the value in the \pcode{integer} domain to the ``FizzBuzz domain''. Hmm!--\emph{mapping} and \emph{domain} sound like mathematics; and functional programming is supposed to be somehow more mathematical. And mathematics is jolly wonderful. 

The first step in making the \pcode{fizzBuzz} more mathematical is to make it map an input to exactly one useful output. Right now, its return type now is \pcode{Unit}, which is a bit like \pcode{void} in Java and C; changing its definition to \pcode{def fizzBuzz2(max: Int): Unit} (and then using the \pcode{max} parameter in the loop) isn't particularly useful: it is a mapping from a number to \pcode{Unit}. And, if this were mathematics, there can be only one such mapping: \pcode{def fizzBuzz2(max: Int): Unit = ()}. Instead of printing the elements to the console, the implementation needs to return a value that can be printed. A a simple \pcode{String} would do, but a \pcode{Seq} of \pcode{String}s is better. The type becomes \pcode{Int => Seq[String]}, and the implementation is shown in \autoref{code:fb2}.

\begin{lstlisting}[caption={Fizz Buzz}, label={code:fb2}, language=Scala, escapechar=|]
def fizzBuzz(max: Int): Seq[String] = {
  var result = List.empty[String]
  for (i <- 1 to max) {
    if (i % 15 == 0) result = result :+ "FizzBuzz"
    else if (i % 3 == 0) result = result :+ "Fizz"
    else if (i % 5 == 0) result = result :+ "Buzz"
    else result = result :+ i.toString
  }
  result
}
\end{lstlisting}

This is a huge improvement! The \pcode{fizzBuzz} is now indeed a function: it maps input to output and its result depends only on the value of the parameter. It would even be possible to pre-compute the result for all possible values of the input and replace the function's body with a look-up in that table: the function would become just data! 

Well, the outside looks great, but the implementation stinks! It uses mutation, and what about the strange \pcode{:+} operator in \pcode{result :+ "Fizz"}, never mind the \pcode{for (i <- 1 to max) \{...\}} nonsense!

\begin{lstlisting}[caption={Fizz Buzz}, label={code:fb3}, language=Scala, escapechar=|]
def fizzBuzz(max: Int): Seq[String] = {
  def fb(i: Int): String =
    if (i % 15 == 0) "FizzBuzz"
    else if (i % 3 == 0) "Fizz"
    else if (i % 5 == 0) "Buzz"
    else i.toString

  (1 to max).map(fb)
}
\end{lstlisting}




In Scala, every concrete type (except \pcode{Nothing}) can have a value: for example, the type \pcode{Boolean} is inhabited by values \pcode{true, false}; the type \pcode{Int} is inhabited by values such as \pcode{5, 42, -100, 0, ...}; the type \pcode{String} is inhabited by values such as \pcode{"Hi", ":)", ""}; the type \pcode{Unit} is inhabited by the only value \pcode{()}. (No, really, it's perfectly good Scala syntax to write \pcode{()} as value. It's just not particularly useful.) The only type that does not have any inhabitants is \pcode{Nothing}: it represents expressions that \emph{diverge}, for example throwing an exception.

Taking a more precise look at \pcode{def fizzBuzz} reveals its type to be \pcode{Unit}; it evaluates to only one value, namely \pcode{()}. If it were a function in the sense of strictly mapping input to output, it would be no different from any other \pcode{()} \emph{constant}. But \pcode{fizzBuzz} does some additional work before returning \pcode{()}; this additional work is not represented by its type, even though it is its \emph{raison d'être}. 

In Java and C, there is no \emph{value} of type \pcode{void}

 As it stands, its type is \pcode{() => Unit}, 

\newpage

\section{Pattern matching}
sasd
\newpage

\section{}

\printbibliography

\end{document}
